\documentclass{chi-ext}
% Please be sure that you have the dependencies (i.e., additional LaTeX packages) to compile this example.
% See http://personales.upv.es/luileito/chiext/

\copyrightinfo{
  Copyright is held by the author/owner(s).\\
  \emph{CHI'13}, April 27 -- May 2, 2013, Paris, France.\\
  ACM 978-1-XXXX-XXXX-X/XX/XX.\\
}

\title{PaintMobile3D: A Novel Android Application to Draw in 3D}

\numberofauthors{4}
% Notice how author names are alternately typesetted to appear ordered in 2-column format;
% i.e., the first 4 autors on the first column and the other 4 auhors on the second column.
% Actually, it's up to you to strictly adhere to this author notation.
\author{
  \alignauthor{
	\textbf{Nathan Jordan}\\
	\affaddr{University of Nevada, Reno}\\
	\affaddr{1664 North Virginia Street  Reno, NV 89557 USA}\\
	\email{njordan@cse.unr.edu}
  }\alignauthor{
  	\textbf{Halim Ates}\\
  	\affaddr{University of Nevada, Reno}\\
  	\affaddr{1664 North Virginia Street  Reno, NV 89557 USA}\\
  	\email{hcagri@gmail.com} } \\ \\ \\
  \alignauthor{
  	\textbf{Thomas Kelly}\\
  	\affaddr{University of Nevada, Reno}\\
  	\affaddr{1664 North Virginia Street  Reno, NV 89557 USA}\\
  	\email{tjbk123@gmail.com}
  }\alignauthor{
  	\textbf{Dr. Sergiu Dascalu}\\
  	\affaddr{University of Nevada, Reno}\\
  	\affaddr{1664 North Virginia Street  Reno, NV 89557 USA}\\
  	\email{dascalus@cse.unr.edu}
  }
}

\teaser{
  \includegraphics[width=\columnwidth]{teaser2.png}
  \label{fig:teaser}
}

% Paper metadata (use plain text, for PDF inclusion and later re-using, if desired)
\def\plaintitle{PaintMobile3D: A Novel Android Application to Draw in 3D}
\def\plainauthor{Nathan Jordan, Thomas Kelly, Halim Ates, Dr. Sergiu Dascalu}
\def\plainkeywords{smartphones, android, drawing, computer vision }
\def\plaingeneralterms{Design, Algorithms, Human Factors}

\hypersetup{
  % Your metadata go here
  pdftitle={\plaintitle},
  pdfauthor={\plainauthor},  
  pdfkeywords={\plainkeywords},
  pdfsubject={\plaingeneralterms},
  % Quick access to color overriding:
  citecolor=black,
  linkcolor=blue,
  menucolor=black,
  urlcolor=blue,
}

\usepackage{graphicx}   % for EPS use the graphics package instead
\usepackage{balance}    % useful for balancing the last columns
\usepackage{bibspacing} % save vertical space in references
%\usepackage{natbib}

\begin{document}

\maketitle

\begin{abstract}
Traditional 3D painting software has relied on building a two-dimensional representation of three-dimensional space in which the user can rotate, pan, and zoom. PaintMobile3D takes a different approach. As smartphones become faster and more feature rich, we can take advantage of these improvements to do things that would previously be infeasible. Using a phone's camera, accelerometer, gyroscope, and/or compass, PaintMobile3D will be able to track the users’ movements as they move the phone around in the real world, allowing them to draw in 3D, based on the movements of the device in the real world. The user will be able to pick from different colors and can make multiple ``strokes'' using the phone to create more advanced drawings.
\end{abstract}

\keywords{\plainkeywords}

\category{H.5.2}{Information interfaces and presentation (e.g., HCI)}{User Interfaces}. 
%See \cite{ACMCCS} 
%See: \url{http://www.acm.org/about/class/1998/} 
%\textcolor{red}{Mandatory section to be included in your final version.}

\terms{\plaingeneralterms}


% =============================================================================
\section{Introduction}
% =============================================================================

Lorem ipsum dolor sit amet, consectetur adipisicing elit, sed do eiusmod tempor incididunt ut labore et dolore magna aliqua. Ut enim ad minim veniam, quis nostrud exercitation ullamco laboris nisi ut aliquip ex ea commodo consequat. Duis aute irure dolor in reprehenderit in voluptate velit esse cillum dolore eu fugiat nulla pariatur. Excepteur sint occaecat cupidatat non proident, sunt in culpa qui officia deserunt mollit anim id est laborum. Lorem ipsum dolor sit amet, consectetur adipisicing elit, sed do eiusmod tempor incididunt ut labore et dolore magna aliqua. Ut enim ad minim veniam, quis nostrud exercitation ullamco laboris nisi ut aliquip ex ea commodo consequat. Duis aute irure dolor in reprehenderit in voluptate velit esse cillum dolore eu fugiat nulla pariatur. Excepteur sint occaecat cupidatat non proident, sunt in culpa qui officia deserunt mollit anim id est laborum.

Lorem ipsum dolor sit amet, consectetur adipisicing elit, sed do eiusmod tempor incididunt ut labore et dolore magna aliqua. Ut enim ad minim veniam, quis nostrud exercitation ullamco laboris nisi ut aliquip ex ea commodo consequat. Duis aute irure dolor in reprehenderit in voluptate velit esse cillum dolore eu fugiat nulla pariatur. Excepteur sint occaecat cupidatat non proident, sunt in culpa qui officia deserunt mollit anim id est laborum. Lorem ipsum dolor sit amet, consectetur adipisicing elit, sed do eiusmod tempor incididunt ut labore et dolore magna aliqua. Ut enim ad minim veniam, quis nostrud exercitation ullamco laboris nisi ut aliquip ex ea commodo consequat. Duis aute irure dolor in reprehenderit in voluptate velit esse cillum dolore eu fugiat nulla pariatur. Excepteur sint occaecat cupidatat non proident, sunt in culpa qui officia deserunt mollit anim id est laborum.

\pagebreak

\begin{figure}
%\hspace*{-0.4\columnwidth}% displace figure
\parbox{\columnwidth}{
  \centering
  \includegraphics[width=0.8\columnwidth]{onex.jpg}
  \caption{The main interface the user is presented with}
  \label{fig:maininterface}
}
\end{figure}

\pagebreak

% =============================================================================
\section{Copyright}
% =============================================================================
Copyright (c) 2012, University of Nevada, Reno Department of Computer Science.
All rights reserved.

\section{Intended Users \& Usability Goals}

PaintMobile3D is an end-user product that will largely be used for the users enjoyment. The application is currently built exclusively for the Android platform, as such the intended users of the application will be Android smartphone users with a device running Android 2.3.x (Gingerbread) or newer. The application currently does not have a specific problem domain outside of entertainment, and as such can be used by any user without any prior training. PaintMobile3D requires no configuration for most users, minimal user interface elements (two options on the ActionBar), and a simple command structure (hold screen to draw, release to stop drawing). Consequently, PaintMobile3D is a very simple and easy to use application. Our main goals when creating PaintMobile3D are the following:

\begin{itemize}
\item
Create a unique drawing tool for the Android platform
\item
Combine the accelerometer, gyroscope and camera data
\item
Allow the user to draw by moving their phone through space
\end{itemize}

\section{Main Functionality}

The core functionality of PaintMobile3D revolves around the ability to paint in three-dimensional space. This functionality is provided by combining the sensor data of the camera of the device as well as the accelerometer and gyroscope (if available), which are combined using an algorithm to provide device localization (the location of the phone in a three-dimensional space). We use these measurements to create points in a virtual coordinate system that can then be used to draw points, lines, or other geometric shapes. Once the user has drawn his or her object in the virtual 3D environment, they are free view the drawing from different angles and perspectives by orbiting, panning, or zooming the drawing. These operations are performed by making intuitive gestures on the screen of the device, such as a swipe to orbit or pinch to zoom.

\section{Related Systems \& Novelty}

At the time of writing, there are no competing or related systems for drawing in 3D space using a smartphone. Similar systems exist using camera-based tracking algorithms to track hand or object movement from a fixed point away from the user. One such solution takes advantage of the Microsoft Kinect stereoscopic camera to measure the movement of the user’s hand and deduce 3D measurements from it to create a single line segment. Another solution involves instrumentation attached to the users hand that tracks their movements and allows them to create strokes in 3D space. Although these systems allow for tracking an object in 3D space, they do so using systems that are unavailable on most smartphones, or do not provide flexibility in regards to location or portability. PaintMobile3D uses the limited sensory data provided by a modern smartphone to perform these operations in a multitude of environments and conditions. The application also provides features like sharing and social network integration that are unavailable on these related systems. Given the growing ubiquity of smartphones and social media outlets like Facebook, Twitter, and Instagram, PaintMobile3D could create a new class of social interaction with 3D art. \cite{brooks1997determining} \cite{schkolne2002drawing} \cite{voigt2011robust} \cite{hol2007robust}

\begin{figure}
%\hspace*{-0.4\columnwidth}% displace figure
\parbox{\columnwidth}{
  \centering
  \includegraphics[width=\columnwidth]{icon.jpg}
  \caption{The program is launched via the main Android launcher or a homescreen shortcut}
  \label{fig:icon}
}
\end{figure}

\begin{figure}
\hspace{\columnwidth}% displace figure
\parbox{\columnwidth}{
  \centering
  \includegraphics[width=\columnwidth]{colorpicker.jpg}
  \caption{When the user selects the eyedropper option in the Action Bar, the user is presented with a popup interface to select a color they wish to draw with.}
  \label{fig:colorpicker}
}
\end{figure}

%\begin{figure}
%\hspace*{-0.4\columnwidth}% displace figure
%\parbox{1.4\columnwidth}{
%  \centering
%  \includegraphics[width=1.4\columnwidth]{sample.jpg}
%  \caption{Insert a caption below each figure. Images can "float" around body text, like this example.}
%  \label{fig:sample}
%}
%\end{figure}

\section{User Interface}

Lorem ipsum dolor sit amet, consectetur adipisicing elit, sed do eiusmod tempor incididunt ut labore et dolore magna aliqua. Ut enim ad minim veniam, quis nostrud exercitation ullamco laboris nisi ut aliquip ex ea commodo consequat. Duis aute irure dolor in reprehenderit in voluptate velit esse cillum dolore eu fugiat nulla pariatur. Excepteur sint occaecat cupidatat non proident, sunt in culpa qui officia deserunt mollit anim id est laborum. Lorem ipsum dolor sit amet, consectetur adipisicing elit, sed do eiusmod tempor incididunt ut labore et dolore magna aliqua. Ut enim ad minim veniam, quis nostrud exercitation ullamco laboris nisi ut aliquip ex ea commodo consequat. Duis aute irure dolor in reprehenderit in voluptate velit esse cillum dolore eu fugiat nulla pariatur. Excepteur sint occaecat cupidatat non proident, sunt in culpa qui officia deserunt mollit anim id est laborum.Lorem ipsum dolor sit amet, consectetur adipisicing elit, sed do eiusmod tempor incididunt ut labore et dolore magna aliqua. Ut enim ad minim veniam, quis nostrud exercitation ullamco laboris nisi ut aliquip ex ea commodo consequat. Duis aute irure dolor in reprehenderit in voluptate velit esse cillum dolore eu fugiat nulla pariatur. Excepteur sint occaecat cupidatat non proident, sunt in culpa qui officia deserunt mollit anim id est laborum. Lorem ipsum dolor sit amet, consectetur adipisicing elit, sed do eiusmod tempor incididunt ut labore et dolore magna aliqua. Ut enim ad minim veniam, quis nostrud exercitation ullamco laboris nisi ut aliquip ex ea commodo consequat. Duis aute irure dolor in reprehenderit in voluptate velit esse cillum dolore eu fugiat nulla pariatur. Excepteur sint occaecat cupidatat non proident, sunt in culpa qui officia deserunt mollit anim id est laborum.

Lorem ipsum dolor sit amet, consectetur adipisicing elit, sed do eiusmod tempor incididunt ut labore et dolore magna aliqua. Ut enim ad minim veniam, quis nostrud exercitation ullamco laboris nisi ut aliquip ex ea commodo consequat. Duis aute irure dolor in reprehenderit in voluptate velit esse cillum dolore eu fugiat nulla pariatur. Excepteur sint occaecat cupidatat non proident, sunt in culpa qui officia deserunt mollit anim id est laborum. Lorem ipsum dolor sit amet, consectetur adipisicing elit, sed do eiusmod tempor incididunt ut labore et dolore magna aliqua. Ut enim ad minim veniam, quis nostrud exercitation ullamco laboris nisi ut aliquip ex ea commodo consequat. Duis aute irure dolor in reprehenderit in voluptate velit esse cillum dolore eu fugiat nulla pariatur. Excepteur sint occaecat cupidatat non proident, sunt in culpa qui officia deserunt mollit anim id est laborum.


\section{Technology \& Implementation}

Since PaintMobile3D is an Android application, we naturally have made heavy use of the Android API. Sensor data is aquired using a publisher-subsciber system that is made available by the Android operating system. In order to draw the points onto the display for the user to see, we have made use of the OpenGL ES library that is also provided by the Android API. While the accelerometer and gyroscope provide directly applicable information straight from the sensor, the data from the camera requires additional manipulation in order to make it useful. For this, are using the OpenCV implementation for Android, called OpenCV4Android to provide with computer vision functionality such as feature extraction and motion tracking. After this data has been processed, it can then be used in our localization algorithms. Some of the calculations needed to perform localization for our application are computationally taxing. Since the application is designed to run on a mobile device that has limited battery power, we need to make these computations as efficient as possible. To do this, we are using the Android NDK (Native Development Kit) to provide us with direct access to the CPU via a C interface (as opposed to the Dalvic Virtual Machine used by Java components of the application).

\section{Potential Applications}

PaintMobile3D is an application that is designed simply for user entertainment. However the inner-workings of the application, coupled with the mobility and popularity of smartphones can be applied to more productive means as well.

\section{Acknowledgements}

Lorem ipsum dolor sit amet, consectetur adipisicing elit, sed do eiusmod tempor incididunt ut labore et dolore magna aliqua. Ut enim ad minim veniam, quis nostrud exercitation ullamco laboris nisi ut aliquip ex ea commodo consequat. Duis aute irure dolor in reprehenderit in voluptate velit esse cillum dolore eu fugiat nulla pariatur. Excepteur sint occaecat cupidatat non proident, sunt in culpa qui officia deserunt mollit anim id est laborum. Lorem ipsum dolor sit amet, consectetur adipisicing elit, sed do eiusmod tempor incididunt ut labore et dolore magna aliqua. Ut enim ad minim veniam, quis nostrud exercitation ullamco laboris nisi ut aliquip ex ea commodo consequat. Duis aute irure dolor in reprehenderit in voluptate velit esse cillum dolore eu fugiat nulla pariatur. Excepteur sint occaecat cupidatat non proident, sunt in culpa qui officia deserunt mollit anim id est laborum.


\balance

\bibliographystyle{acm-sigchi}
\bibliography{references}

\end{document}
